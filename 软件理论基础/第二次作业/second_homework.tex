\documentclass[11pt,oneside,a4paper]{article}
\author{黄国鹏}
\usepackage{ctex}
\usepackage{graphicx}
\usepackage{tabularray}
\usepackage{tagging}
\usepackage{amsthm}
\usepackage{amsmath}
\title{软件理论基础第二次作业 }
\begin{document}
\maketitle
\begin{itemize}
    \item[1.] 设$A \downarrow B $表示$\neg (A \vee B)$,证明连接符$\{ \downarrow \}$是命题逻辑连接符的充足集。
        
        \textbf{证明:}

            由数学归纳法:若$\lvert F \vert = 1$,即$F$为原子公式。则必为一个原子命题或其否定形式,记原子命题为$p$,则$F = p$或$F = \neg P $成立,则有:
            \begin{enumerate}
                \item[(1)]\[F = p = \neg \neg (p \vee p) = \neg (p \downarrow p) = \neg ((p \downarrow {\rm{p}}) \vee (p \downarrow p)) = ({\rm{p}} \downarrow p) \downarrow (p \downarrow p)\]
                \item[(2)]\[F = \neg p = \neg (p \vee p) = p \downarrow p\]
            \end{enumerate}
            假设对于所有的公式$F$,若$\lvert F \vert < n$则$F$能用'$\downarrow$'表示,现考虑$\lvert F \vert = n$
            \begin{enumerate}
                \item[(1)]若$F=\neg A$ ($A$为 $F$的子公式),可得: 
                \[F = \neg A = \neg (A \vee A){\rm{ }} = A \downarrow A.\]
                \item[(2)]若$F = A \vee B$ ($A$,$B$为F的子公式),可得:
                \[F = A \vee B = \neg (\neg (A \vee B)){\rm{ }} = \neg (A \downarrow B){\rm{ }} = {\rm{ }}(A \downarrow B) \downarrow (A \downarrow B)\]
            \end{enumerate}
            因此,无论何种情况A、B必能由'$\downarrow$'表示,所以由数学归纳法可得所有公式$F$均能用'$\downarrow$'表示,即$\left\{  \downarrow  \right\}$ 是充足集。
    \item[2.] 
        \begin{itemize}
            \item[(1)] 证明\[ \vdash (B \to C) \to ((A \to B) \to (B \to C))\] \par
            \textbf{证明:} \quad 由演绎定理,只需证明
                            \begin{equation*}
                               \{ (B \to C) \}  \vdash ((A \to B) \to (B \to C))
                            \end{equation*}
            \begin{itemize}
                \item[(1)]  
                            \begin{equation*}
                                B \to C                      \tag*{$\varGamma$}
                            \end{equation*}
                \item[(2)] 
                            \begin{equation*}
                               (B \to C) \to ((A \to B) \to (B \to C))  \tag*{L2}
                            \end{equation*}
                \item[(3)]
                            \begin{equation*}
                                (A \to B) \to (B \to C)      \tag*{MP(1,2)}
                            \end{equation*}
            \end{itemize}
            \item[(2)] 证明\[ \vdash (A \to (A \to B)) \to (A \to B)\] \par
            \textbf{证明:} \quad 由演绎定理,只需证明
                        \begin{equation*}
                            \{ A \to (A \to B),A \} \vdash B
                        \end{equation*}
            构造推演序列如下:            
            \begin{itemize}    
                \item[(1)] \begin{equation*}
                             A  \tag*{$\varGamma$}
                            \end{equation*}
                \item[(2)] 
                           \begin{equation*}
                            A \to (A \to B) \tag*{$\varGamma$}
                           \end{equation*}
                \item[(3)]
                           \begin{equation*}
                            A \to B \tag*{MP(1,2)}
                           \end{equation*}  
                \item[(4)]
                           \begin{equation*}
                               B \tag*{MP(1,3)}
                           \end{equation*}  
            \end{itemize}
            
        \end{itemize}
    \item[3.] 试证:
        \begin{itemize}
            \item[(1)]   证明\[(A \to (B \to C)) \approx (B \to (A \to C))\]
            \textbf{证明} 
            \begin{itemize}
                \item[1.] \[  \vdash  (A \to (B \to C)) \to (B \to (A \to C)) \] \par
                    由演绎定理,只需证明 
                                \[ \{A \to (B \to C),B\} \vdash (A \to C) \]
                    构造推演序列如下:
                    \begin{enumerate}
                        \item[(1)] 
                                \begin{equation*}
                                     A \to (B \to C)             \tag*{$\varGamma$}
                                \end{equation*}                    
                        \item[(2)] 
                                \begin{equation*}
                                    (A \to (B \to C)) \to ((A \to B) \to (A \to C)) \tag*{L2}
                                \end{equation*}
                        \item[(3)]
                                \begin{equation*}
                                   (A \to B) \to (A \to C)      \tag*{MP(1,2)}
                                \end{equation*}
                        \item[(4)]
                                \begin{equation*}
                                    B                           \tag*{$\varGamma$}
                                \end{equation*}
                        \item[(5)]
                                \begin{equation*}
                                        B \to (A \to B)         \tag*{L1}
                                \end{equation*}
                        \item[(6)]
                                \begin{equation*}
                                        A \to B         \tag*{MP(4,5)}
                                \end{equation*}
                        \item[(7)]
                                \begin{equation*}
                                        A \to C         \tag*{MP(3,6)}
                                \end{equation*}                               
                    \end{enumerate}
                    故$\{A \to (B \to C),B\} \vdash (A \to C)$,即$\vdash (A \to (B \to C)) \to (B \to (A \to C))$
                
                \item[2.] \[ \vdash (B \to (A \to C)) \to (A \to (B \to C)) \] \par
                    由演绎定理,只需证明 
                                \[ \{B \to (A \to C),A\} \vdash (B \to C) \]
                    构造推演序列如下:
                    \begin{enumerate}
                        \item[(1)] 
                                \begin{equation*}
                                     B \to (A \to C)             \tag*{$\varGamma$}
                                \end{equation*}                    
                        \item[(2)] 
                                \begin{equation*}
                                    (B \to (A \to C)) \to ((B \to A) \to (B \to C)) \tag*{L2}
                                \end{equation*}
                        \item[(3)]
                                \begin{equation*}
                                   (B \to A) \to (B \to C)      \tag*{MP(1,2)}
                                \end{equation*}
                        \item[(4)]
                                \begin{equation*}
                                    A                           \tag*{$\varGamma$}
                                \end{equation*}
                        \item[(5)]
                                \begin{equation*}
                                        A \to (B \to A)         \tag*{L1}
                                \end{equation*}
                        \item[(6)]
                                \begin{equation*}
                                        B \to A         \tag*{MP(4,5)}
                                \end{equation*}
                        \item[(7)]
                                \begin{equation*}
                                        B \to C         \tag*{MP(3,6)}
                                \end{equation*}                               
                    \end{enumerate}
                    故$\{B \to (A \to C),A\} \vdash (B \to C)$,即$\vdash  (B \to (A \to C)) \to (A \to (B \to C))$   
            \end{itemize}
            因此,\[(A \to (B \to C)) \approx (B \to (A \to C))\]

            \item[(2)]   证明\[(A \to (A \to B)) \approx (A \to B)\]
            \textbf{证明}
            \begin{itemize}
                \item[1.]   \[\{ A \to (A \to B)\}  \vdash (A \to B)\]
                    由演绎定理,只需证明
                            \[\{ A \to (A \to B),A\}  \vdash (B)\]
                    \begin{enumerate}
                        \item[(1)]
                                \begin{equation*}
                                    A \tag*{$\varGamma$}
                                \end{equation*}
                        \item[(2)]
                                \begin{equation*}
                                    A \to (A \to B) \tag*{$\varGamma$}
                                \end{equation*}
                        \item[(3)]
                                \begin{equation*}
                                    A \to B \tag*{MP(1,2)}
                                \end{equation*}
                        \item[(4)]
                                \begin{equation*}
                                   B \tag*{MP(1,3)}
                                \end{equation*}
                    \end{enumerate}
                    故\[\{ A \to (A \to B),A\}  \vdash (B)\]
                    即\[ \vdash (A \to (A \to B)) \to (A \to B)\]
                \item[2.]   \[\{ A \to B \}  \vdash  (A \to (A \to B)) \]
                    \begin{enumerate}
                        \item[(1)]
                                \begin{equation*}
                                    A \to B                         \tag*{$\varGamma$}
                                \end{equation*}
                        \item[(2)]
                                \begin{equation*}
                                    (A \to B) \to (A \to (A \to B)) \tag*{L1}
                                \end{equation*}
                        \item[(3)]
                                \begin{equation*}
                                    A \to (A \to B)                 \tag*{MP(1,2)}
                                \end{equation*}
                    \end{enumerate}
                    故\[\{ A \to B \}  \vdash  (A \to (A \to B)) \]
                    即\[ \vdash (A \to B) \to (A \to (A \to B))\]
                

            \end{itemize}
            因此,\[(A \to (A \to B)) \approx (A \to B)\]
        \end{itemize}
\end{itemize}

\end{document}