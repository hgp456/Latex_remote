\documentclass[11pt,oneside,a4paper]{article}
\author{黄国鹏}
\usepackage{ctex}
\usepackage{graphicx}
\usepackage{tabularray}

\title{软件理论基础第一次作业 }
\begin{document}
\maketitle
    \begin{itemize}
        \item[1-(1)]
        题目:证重言式\[(A \to (B \to C)) \to ((A \to B) \to (A \to C))\] \par
        解:列出真值表
        \begin{table}[!h]
        \centering
            \begin{tabular}{|l|l|l|c|}
            \hline
            A  & B  & C  & $(A \to (B \to C)) \to ((A \to B) \to (A \to C))$ \\ \hline
            0  & 0 &  0  & 1 \\ \hline
            0  & 0 &  1  & 1\\ \hline
            0  & 1 &  0  & 1 \\ \hline
            0  & 1 &  1  & 1 \\ \hline
            1  & 0 &  0  & 1 \\ \hline
            1  & 0 &  1  & 1 \\ \hline
            1  & 1 &  0  & 1 \\ \hline
            1  & 1 &  1  & 1 \\ \hline
            \end{tabular}
            \end{table}
        由真值表可得公式为永真式
        \item[1-(2)] 
        题目:证重言式\[(\neg A \to \neg B) \to (B \to A)\]
        解:
        \[\begin{array}{l}
            (\neg A \to \neg B) \to (B \to A)\\
             = (\neg \neg A \vee \neg B) \to (B \to A)\\
             = (A \vee \neg B) \to (B \to A)\\
             = (\neg B \vee A) \to (B \to A)\\
             = (B \to A) \to (B \to A)\\
             = \neg (B \to A) \vee (B \to A)\\
             = 1
            \end{array}\]
        因此此公式为永真式
        \item[2-(1)]
        题目:  \[(A \wedge B) \to C = (A \to C) \vee (B \to C)\] \par 
        解:
    \[\begin{array}{l}
    leftform = (A \wedge B) \to C\\
    = \neg (A \wedge B) \vee C\\
    = \neg A \vee \neg B \vee C\\
    = \neg A \vee C \vee \neg B \vee C\\
    = \neg A \vee C \vee \neg B \vee C\\
    = (A \to C) \vee (B \to C) \\
    = rightform
   \end{array}\]
   由上述推导,左式等于右式,因此公式为永真式
        \item[2-(2)] 
        题目:\[(A \to (B \to C) = B \to (A \to C)\]
        解:
        \[\begin{array}{l}
            leftfrom = (A \to (B \to C)\\
            = A \to (\neg B \vee C)\\
            = \neg A \vee (\neg B \vee C)\\
            = \neg B \vee (\neg A \vee C)\\
            = B \to (\neg A \vee C)\\
            = B \to (A \to C) \\
            = rightform
           \end{array}\]
    由上述推导,左式等于右式,因此公式为永真式
    \newpage
        \item[3] $(\neg \mathop p\nolimits_1  \to {\rm{ }}\mathop p\nolimits_2 {\rm{ }}){\rm{ }} \to \mathop p\nolimits_3$ 的析取范式和合取范式 \par
        解:
        \begin{table}[!h]
        \centering
            \begin{tabular}{|l|l|l|c|}
            \hline
            $p_1$ & $p_2$ & $p_3$ & $(\neg \mathop p\nolimits_1  \to {\rm{ }}\mathop p\nolimits_2 {\rm{ }}){\rm{ }} \to \mathop p\nolimits_3$ \\ \hline
            0  & 0  & 0  & 1  \\ \hline
            0  & 0  & 1  & 1  \\ \hline
            0  & 1  & 0  & 0  \\ \hline
            0  & 1  & 1  & 1  \\ \hline
            1  & 0  & 0  & 0  \\ \hline
            1  & 0  & 1  & 0  \\ \hline
            1  & 1  & 0  & 0  \\ \hline
            1  & 1  & 1  & 1  \\ \hline
            \end{tabular}
            \end{table}
        
        \par
        由真值表易得\par
        主析取范式为:
        \[(\neg \mathop p\nolimits_1  \wedge {\rm{ }}\neg \mathop p\nolimits_2 {\rm{ }}\mathop { \wedge \neg p}\nolimits_3 ) \vee (\neg \mathop p\nolimits_1  \wedge {\rm{ }}\mathop {\neg p}\nolimits_2 {\rm{ }}\mathop { \wedge p}\nolimits_3 ) \vee (\neg \mathop p\nolimits_1  \wedge {\rm{ }}\mathop p\nolimits_2 {\rm{ }}\mathop { \wedge p}\nolimits_3 ) \vee (\mathop p\nolimits_1  \wedge {\rm{ }}\mathop p\nolimits_2 {\rm{ }}\mathop { \wedge p}\nolimits_3 )\]
        主合取范式为:
        \[(\mathop p\nolimits_1  \vee {\rm{ }}\neg \mathop p\nolimits_2 {\rm{ }} \vee \mathop p\nolimits_3 ) \wedge (\neg \mathop p\nolimits_1  \vee {\rm{ }}\mathop p\nolimits_2 {\rm{ }} \vee \mathop p\nolimits_3 ) \wedge (\mathop {\neg p}\nolimits_1  \vee {\rm{ }}\mathop p\nolimits_2  \vee \mathop p\nolimits_3 ) \wedge (\neg \mathop p\nolimits_1  \vee {\rm{ }}\neg \mathop p\nolimits_2 {\rm{ }} \vee \mathop p\nolimits_3 )\]
        \item[4-(1)] 
        题目:\[\left( {\mathop p\nolimits_1  \vee \mathop p\nolimits_2 } \right) \to \mathop p\nolimits_3 \]
        解:\[\tau (\left( {{\rm{ }}{p_1} \vee {\rm{ }}{p_2}} \right) \to {\rm{ }}{p_3}) = \frac{5}{{\mathop 2\nolimits^3 }} = \frac{5}{8}\]
        \item[4-(2)]
        题目:\[\left( {\mathop p\nolimits_1  \to \mathop p\nolimits_2 } \right) \vee \left( {\mathop p\nolimits_3  \to \mathop p\nolimits_4 } \right)\]
        解:\[\tau (\left( {{\rm{ }}{p_1} \to {\rm{ }}{p_2}} \right) \vee \left( {{\rm{ }}{p_3} \to {\rm{ }}{p_4}} \right)) = \frac{{15}}{{\mathop 2\nolimits^4 }} = \frac{{15}}{{16}}\]
        \item[4-(3)]
        题目:\[(\neg \mathop p\nolimits_1  \to {\rm{ }}\mathop p\nolimits_2 ) \to {\rm{ }}\mathop p\nolimits_3 \]
        解: \[\tau ((\neg {\rm{ }}{p_1} \to {\rm{ }}{p_2}) \to {\rm{ }}{p_3}) = \frac{5}{{\mathop 2\nolimits^3 }} = \frac{{5}}{8}\]
    \end{itemize}

\end{document}